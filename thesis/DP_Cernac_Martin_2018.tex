% hidelinks remove colour boxes around hyperlinks
\documentclass[thesis=M,czech]{FITthesis}[2012/10/20]

\usepackage[utf8]{inputenc} % LaTeX source encoded as UTF-8
\usepackage{graphicx} %graphics files inclusion
\usepackage{amsmath} %advanced maths
\usepackage{amssymb} %additional math symbols
\usepackage{dirtree}
\usepackage{pdfpages}
\usepackage{listings}
\lstset{
basicstyle=\small\ttfamily,
columns=flexible,
breaklines=true,
captionpos=b
}
\renewcommand{\lstlistingname}{Ukázka kódu}

% % list of acronyms
% \usepackage[acronym,nonumberlist,toc,numberedsection=autolabel]{glossaries}
% \iflanguage{czech}{\renewcommand*{\acronymname}{Seznam pou{\v z}it{\' y}ch zkratek}}{}
% \makeglossaries

\department{Katedra počítačových systémů}
\title{Komunikace skrze Captive portal}
\authorGN{Martin} %(křestní) jméno (jména) autora
\authorFN{Černáč} %příjmení autora
\authorWithDegrees{Bc. Martin Černáč} %jméno autora včetně akademických titulů
\author{Martin Černáč} %jméno autora bez akademických titulů
\supervisor{Ing. Aleš Padrta, Ph. D.}
\acknowledgements{Rád bych poděkoval svému vedoucímu za cenné rady, věcné připomínky a vstřícnost při konzultacích.}
\abstractCS{TODO V~několika větách shrňte obsah a přínos této práce v~češtině. Po přečtení abstraktu by měl mít čtenář dost informací pro rozhodnutí, zda chce Vaši práci číst.}
\abstractEN{TODO Sem doplňte ekvivalent abstraktu Vaší práce v~angličtině.}
\placeForDeclarationOfAuthenticity{V~Praze}
\declarationOfAuthenticityOption{4} %volba Prohlášení
\keywordsCS{Závěrečná práce, \LaTeX{}.}
\keywordsEN{Thesis, \LaTeX{}.}
\website{https://github.com/octaroot/CTU-FIT-MasterThesis} %volitelná URL práce, objeví se v tiráži


%dobre zdroje
%https://www.secplicity.org/2016/08/26/lessons-defcon-2016-bypassing-captive-portals/
%https://en.wikipedia.org/wiki/Captive_portal
%https://www.ietf.org/mail-archive/web/captive-portals/current/threads.html#00090

%seznam sw reseni
%https://mohammadthalif.wordpress.com/2010/12/14/list-of-open-source-captive-portal-software-and-network-access-control-nac/

%jak captive skodi a jak to delat nejlepe
%https://www.eff.org/deeplinks/2017/08/how-captive-portals-interfere-wireless-security-and-privacy

%ukazkova byznys stranka, ktera vychvaluje svuj captive sw ... lol
%prvni argument je uplne mimo, na podvrzene HTML strance neni nic legitniho, uplna blbost
%treti argument zminuje tracking uzivatele diky vynuceni loginu skrze socialni site, nebo email (chapu to tak, ze tam proste dam adresu a pusti me dal)
%https://www.securedgenetworks.com/blog/why-is-a-captive-portal-important-for-wireless-guest-access

\begin{document}

\begin{introduction}
Bezdrátové sítě se staly zcela běžným prostředkem mezilidské komunikace. Uživatelé bezdrátové sítě mají možnost si navzájem vyměňovat informace a nebýt přitom omezeni kabelovým spojením. Velkým přínosem bezdrátové sítě je tedy zvýšená mobilita uživatelů. Ta vedla k vlně popularity bezdrátových sítí počínaje mobilními telefony, využívajících bezdrátovou síť \texttt{GSM}, až po dnešní chytré spotřebiče a jejich zapojení do \textit{Internet of Things}.

S rostoucími nároky uživatelů prošly rozsáhlým vývojem i bezdrátové sítě (vyšší prostupnost, nižší latence a další aspekty). Mezi dlouhodobě populární a velmi rozšířené typy bezdrátových sítí se řadí technologie \texttt{Wi-Fi}. Jedná se o technologii podporovanou širokým spektrem spotřební elektroniky (například televizory, tiskárny, mobilní telefony nebo počítače). Technologie \texttt{Wi-Fi} využívá bezlicenčním pásmo \texttt{ISM} a díky tomu je provozování vlastní \texttt{Wi-Fi} sítě legislativně nenáročné. Na trhu je navíc dostupná celá řada produktů, zajišťující provoz \texttt{Wi-Fi} sítě.

Z těchto důvodů došlo k velkému rozmachu takzvaných \textit{hotspotů}, tedy veřejně přístupných míst s pokrytím \texttt{Wi-Fi} sítě. Taková \texttt{Wi-Fi} síť je zpravidla veřejně přístupná a uživatelům nabízí přístup do sítě Internet. Ačkoliv je velice snadné začít s provozem \textit{hotspotu}, je nutné dbát na další aspekty provozu takové služby -- zejména právní aspekty.

Uživatelé \textit{hotspotu} by měli být srozuměni s pravidly používání konkrétní sítě, limitovanou odpovědností provozovatele a před začátkem užívání sítě doložit svůj souhlas s pravidly. Provozovatel navíc může mít zájem o některé identifikující informace o uživatelích \textit{hotspotu}.

Technologie \texttt{Wi-Fi} však sama o sobě neumožňuje nic z výše uvedeného. Takovou situaci lze vyřešit například zapojením recepce v prostředí hotelu (uživatel písemně vyjádří souhlas s pravidly používání sítě, recepční vydá přístupové údaje do sítě). Častěji se však setkáváme s automatizovaným přístupem, realizovaným pomocí \textit{captive portálu} (z angličtiny \textit{Captive portal}).

Řešení s pomocí \textit{captive portálu} spočívá v detekci nově připojených uživatelů, které je nutné informovat o pravidlech provozu sítě. Po udělení souhlasu s pravidly je uživateli poskytnut přístup do Internetu a všechny následné interakce uživatele se sítí \textit{captive portál} ignoruje (nezasahuje do nich).

Z principu věci tedy \textit{captive portál} musí být schopen \textbf{nejprve zasahovat do veškerého síťového provozu} (uživatel doposud nedal souhlas s pravidly, neměl by mít možnost síť využívat) a \textbf{následně do provozu konkrétního uživatele nezasahovat vůbec}. Existuje celá řada technologických postupů pro docílení popsaného efektu. Mnohé z nich jsou však neefektivní a nepočítají s \uv{neposlušným} uživatelem, který se bude snažit omezující techniky překonat.

Právě proto jsem se rozhodl vypracovat diplomovou práci na téma obcházení \textit{captive portálu}, zdůrazňující jejich technologickou nedokonalost a poukázat na lepší řešení řízení síťového přístupu (\textit{Network Access Control}).

V této práci se proto budu zabývat popisem problematiky \textit{captive portálů} a obecnými způsoby jejich obcházení. Jako demonstraci technologické nedokonalosti užití \textit{captive portálu} pro zajištění řízení síťového přístupu rovněž navrhnu a implementuji protokol s důrazem na maximální prostupnost. Implementovaný protokol otestuji a provedu srovnání s dostupnými nástroji pro obcházení \textit{captive portálů}.
\end{introduction}

\chapter{Analýza současné situace}

Tato kapitola se věnuje problematice \textit{captive portálů}, motivací jejich nasazení v síti a častými problémy s používáním \textit{captive portálu} jako nástroje pro zajištění řízení síťového přístupu.

\section{Captive portál}

\textit{Captive portál} představuje webovou aplikaci, často nasazovanou na veřejně přístupných sítích. Aplikace má za úkol informovat nově připojené klienty o podmínkách užití sítě a požadovat uživatelův souhlas s jejich dodržováním. Až do momentu souhlasu s podmínkami užití sítě je uživateli odepřen přístup do zbytku sítě. Z toho plyne první část názvu \textit{\textbf{Captive} portál} -- uživatel je \uv{zajatý}, \uv{uvězněný} (v angličtině \textit{captive}).

\subsection{Motivace nasazení}

% proc to mame. legislativa provozu hotspotu. lidi odslouhlasi podminky sluzby, kryju se tim ja jako provozovatel -- nabizim jim sluzbu, ktera neni sofrovana.
% captive portal na dedikovane guest wifina: oddeleni guest trafficu, snazsi omezeni sirky pasma, data collection -- jaci lide se pripojuji na muj hotspot ... muzu napriklad nabidnou prihlaseni pres FB/jinou soc. sit.

\textit{Captive portál} je do provozu sítě často nasazován jako nástroj pro zajištění řízení síťového přístupu. Přístup do sítě je umožněn pouze klientům, kteří splní podmínky přístupu do sítě. Takovou podmínkou může být pouhé vyjádření souhlasu s používáním konkrétní sítě, ale může se jednat i o podmínku složitější, například:

\begin{itemize}
 \item shlédnutí reklamního spotu dle výběru provozovatele
 \item uhrazení poplatku pro přístup do sítě
 \item poskytnutí některých osobních údajů a souhlasu s jejich zpracováním
 \item doložení oprávnění pro přístup do sítě (kód z účtenky, číslo hotelového pokoje, \ldots)
 \item zviditelnění provozovatele pomocí sociálních médií (například Facebook \textit{check-in})
\end{itemize}

Jak plyne z výše uvedeného výčtu, vyjma právních aspektů může být \textit{captive portál} použit i pro shromažďování údajů o uživatelích sítě. Jedním z nástrojů pro takovou činnost je nabízení \uv{přihlášení se} do \textit{captive portálu} pomocí účtu na některé ze sociálních sítí. Pokud uživatel takovou možnost využije, \textit{captive portál} si od sociální sítě vyžádá informace o uživateli, jako například jméno, fotografii, pohlaví nebo datum narození. Po shromažďování takových informací je uživateli poskytnut přístup do zbytku sítě. Provozovatel tedy může uživatele například identifikovat nebo detekovat opakované návštěvy \textit{hotspotu}. Na oplátku je uživateli \uv{zdarma} poskytnut přístup do sítě Internet.


%TODO pridat citaci primo od FB a mozna obrazek
Pro usnadnění nasazení takového řešení nabízí společnost Facebook službu \textit{Facebook Wi-Fi}\cite{facebook_wifi}, cílenou na majitele obchodů. Jedná se o řešení na bázi \textit{captive portálu}, které vyžaduje aby nově připojený uživatel měl konto na sociální síti Facebook. Po připojení na \textit{hotspot} je uživatel vyzván ke sdílení informace o jeho návštěvě obchodu, jehož \textit{hotspot} právě používá (jako protislužbu za poskytnutý přístup do Internetu).

Poněkud méně invazivní motivací pro zavedení \textit{captive portálu} je monetizace \textit{hotspotu}. Například prodejem reklamního místa -- uživatel po připojení do sítě musí shlédnou reklamní spot, nebo vyplnit krátkou anketu. Provozovatel \textit{hotspotu} získá z takové aktivity finanční odměnu a uživateli je odměněn přístupem do sítě Internet.

Některé \textit{captive portály} alternativně umožňují uživateli doložit nárok na přístup do sítě. Například jednorázový kód z účtenky, čímž dokládá útratu v podniku, který \textit{hotspot} provozuje. Nebo číslo hotelového pokoje, čímž dokládá svůj pobyt v hotelu, který zahrnuje (jinak zpoplatněný) přístup do sítě Internet.

\subsection{Realizační technologie}

% jak captive portal funguje.
% jak je mozne ho realizovat (DNS/L3 ICMP/preklad *:80 na muj server -- casto zarizeni po pripojeni zkusi nacist nejakou well-known URL ktera by mela vratit HTTP 204)

Úkolem \textit{captive portálu} je detekovat nově připojené uživatele sítě, omezit jim přístup do sítě a nasměrovat je na webovou aplikaci captive portálu. Po splnění podmínek pro plnohodnotný přístup uživatele do zbytku sítě nesmí \textit{captive portál} do komunikace dále zasahovat (tj. musí \textit{detekovat}, že síťový provoz patří oprávněnému uživateli).

Ačkoliv se jedná o přímočarý cíl, je možné ho dosáhnout celou řadou postupů a technologií. Proto se v praxi setkáváme s velkým počtem různorodých implementací \textit{captive portálu}. Některé z nich jsou dostupné pod svobodnou licencí, jiné jsou součástí placeného produktu a v neposlední řadě existují řešení \textit{na míru} -- a to nejen \textit{na míru} provozovateli, ale rovněž \textit{na míru} konkrétnímu zařízení/hardware.

Z této skutečnosti plyne fakt, že by bylo velice náročné popisovat a srovnávat \textit{všechny} existující implementace \textit{captive portálu}. V této části práce se proto zmiňuji jen o několika vybraných realizačních technologiích, které dostačují pro pochopení práce \textit{captive portálu}.

Přestože efektu \textit{captive portálu} lze s velkou úspěšností docílit pouhým odkloněním HTTP provozu, existují mnohem sofistikovanější varianty, využívající například oddělené VLAN sítě. Obecně však platí, že \textit{captive portál} při své práci může vycházet pouze z informací, které putují po síti. Detekce nově připojených uživatelů a identifikace oprávněných uživatelů je tedy zpravidla založena dvojici identifikátorů:

\begin{itemize}
 \item globálně unikátní \texttt{MAC} adresa zařízení
 \item přidělená \texttt{IP} adresa zařízení
\end{itemize}

\textit{Captive portál} lokálně ukládá informace o autorizovaných uživatelských zařízeních v síti (zaznamenává jejich \texttt{MAC} a \texttt{IP} adresy). Síťový provoz takových zařízení není narušován. Pokud však uživatel využívá zařízení, které \textit{captive portál} na svém seznamu nenalezne, \textit{captive portál} síťový provoz buď zahodí, nebo zmanipuluje takovým způsobem, aby se uživatel dostal na webovou aplikaci \textit{captive portálu} a mohl se identifikovat. Záznamy na seznamu autorizovaných uživatelů sítě zpravidla podléhají periodickému mazání neaktivních uživatelů -- uživatel je tedy nucen se po delší době nečinnosti opakovaně identifikovat \textit{captive portálu}.

Alternativně k periodickému promazávání seznamu autorizovaných klientů může \textit{captive portál} vyžadovat, aby uživatel po celou dobu používání sítě měl v prohlížeči otevřené speciální okno, jehož přítomnost instruuje \textit{captive portál} k přidělení plnohodnotného síťového přístupu.

Ve chvíli, kdy je \textit{captive portál} schopen rozeznat autorizované a neautorizované uživatele, musí rovněž mít možnost neautorizované uživatele nasměrovat na webovou aplikaci \textit{captive portálu}. Takový cíl \textit{captive portál} často naplňuje prováděním \texttt{MITM} útoku na nově připojené uživatele. Například při přístupu neautorizovaného uživatele na libovolnou webovou stránku protokolem \texttt{HTTP} je jeho provoz odkloněn a vrácena odpověď od \textit{captive portálu}, která prohlížeč uživatele nasměruje na webovou aplikaci \textit{captive portálu}. Kromě této techniky uvádím v následující části textu i několik dalších.


%TODO - nasledujici tri sekce by mozna nebylo spatne vice rozvest. Na to se musim zeptat Alese, neni to uplne pointa tehle diplomky, vytvaret captive portal

\subsubsection{\texttt{ICMP} host redirect}

Protokol \texttt{ICMP} specifikuje zprávy, které může směrovač poslat koncové stanici, pokud detekuje, že stanice v rámce své komunikace používá neoptimální síťovou cestu. Je zcela v režii cílové stanice, zda-li si nechá o svém směrování radit od ostatních zařízení v síti. Tato metoda spoléhá na situaci, kdy koncová stanice skutečně upraví svou směrovací tabulku a zanese do ní informace z \texttt{ICMP} \textit{host redirect} zprávy. Právě s tímto úmyslem odesílá \textit{captive portál} \texttt{ICMP} \textit{host redirect} zprávu, když detekuje pokus o spojení uživatele se serverem v Internetu. \texttt{ICMP} zpráva se pokusí cílovou stanici uživatele přesvědčit, že ideální cesta vede skrze server provozující \textit{captive portál}. Koncová stanice upraví své směrování a začne komunikovat se svým protějškem skrze \textit{captive portál}, který díky tomu může komunikaci manipulovat za účelem nasměrování uživatele na webovou aplikaci \textit{captive portálu}.

\subsubsection{\texttt{HTTP} přesměrování}
Při pokusu o přístup na webovou stránku \texttt{www.example.com} je požadavek klienta odkloněn a odpověď na požadavek zaslána přímo z \textit{captive portálu}. V odpovědi je zpravidla využita \texttt{HTTP} hlavička \texttt{302 Found}, která prohlížeč klienta nasměruje na webovou aplikace \textit{captive portálu}, viz Ukázka \ref{verb:http-302}.

\begin{figure}[h]
  \renewcommand{\lstlistingname}{Ukázka}
  \begin{lstlisting}[label=verb:http-302, caption={Ukázka přesměrování HTTP požadavku (zkráceno)},frame=single]
  > GET / HTTP/1.1
  > Host: www.example.com
  >
  < HTTP/1.1 302 Found
  < Location: http://192.168.1.1/captive/
  \end{lstlisting}
\end{figure}


%TODO - mozna dodat tabulku OS a defaultnich voleb pro prijem redirectu? zminit secure_redirect? dat nejakou citaci treba na ubuntu hardening (vypadla na me z googlu ubuntu secure_redirect)

\subsubsection{Podvržení \texttt{DNS}}

\textit{Captive portál} detekuje \texttt{DNS} požadavky klientů. Pokud požadavek patří neautorizovanému klientovi, je mu nazpět zaslána odpověď s \texttt{IP} adresou webové aplikace \textit{captive portálu}. Jedná se o značně nebezpečnou techniku, protože může snadno dojít k otrávení \texttt{DNS} cache klienta. Pro minimalizaci takového vedlejšího efektu je v rámci DNS odpovědi zaslána hodnota \texttt{TTL} rovna 0, která by měla zajistit, že podvržená odpověď nebude do lokální \texttt{DNS} cache uložena. Ukázka evidentního podvržení \texttt{IP} adresy serveru \texttt{google.com} je zachycena v ukázce \ref{verb:dns-mitm}.

\begin{figure}[h]
  \renewcommand{\lstlistingname}{Ukázka}
  \begin{lstlisting}[label=verb:dns-mitm, caption={Ukázka podvržení DNS odpovědi},frame=single]
$ nslookup google.com
Server:		192.168.1.1
Address:	192.168.1.1#53

Non-authoritative answer:
Name:	google.com
Address: 192.168.1.1
  \end{lstlisting}
\end{figure}


\subsection{Technické problémy}

% technicke problemy. je potreba prohlizec
%    ten ale spousta zarizeni proste nema (herni konzole Nintendo DS napr) -- resi se to \texttt{WISPr}-
% https
% transaprency logs ??
% User je casto identifikovan pouze MAC/IP kombem -> sniff -> spoof.
% casto podvrhavane certifikaty, aby bylo mozne dostat uzivatle na loginpage.
% MITM! jak takove siti mohu verit ??

% obcas captive portal blokuje moc a OS nechape, ze je za captivem
% tzn neotevre se normalni OS-native popup (nejaky prohlizec bez session)
% ale clovek to musi udelat v normalnim prohlizeci .. possible data leaks

% jak captive detekuje chrome-os/chromium
% https://www.chromium.org/chromium-os/chromiumos-design-docs/network-portal-detection

\subsection{Netechnické problémy}

% podvrzene ssl certy uci uzivatele klikat na ``ano, ano, verim vsemu, hlavne at jede FB''.
%
% Nutnost splneni podminek pro pristup - napr. zadat email, pravdepodobne za ucelem SPAMovani.
% Nekdy je nutne i shlednout videoreklamu ala youtube/iprima.
% nekdy i vice invazivni a otravne veci
% o tom uz jsem psal, tak jen strucne

\subsection{Alternativy captive portálů}

% jak se to da delat jinak - 802.1x kdyz uz mame ucty .. wifi treba v kavarne kdyz si neco dam, to si umim predstavit.
% eduroam !! funguje naprosto v pohode! stejne vetsina captive protalu ma nekde v pozadi RADIUS
% negativa - ne kazdy klient to podporuje .. ale je 2018, takovych uz neni mnoho

% kdyz to nejde jinak, tak jak alespon lepe -- viz EFF article v sources/
%   -- napr neblokovat detekci captive portalu (HTTP 204 responses obvykle (No Content) -- protoze pak OS vetsinou nabidne nejaky osekany browser, ve kterem se otevre login form, coz snizuje pst. nejakeho leaku dat.

% Minuly rok nekdo porovnaval FW, NACy a jeden captive - captive je samozrejme k nicemu v porovnani s NACy
% https://dspace.cvut.cz/bitstream/handle/10467/69572/F8-BP-2017-Lauer-Ondrej-thesis.pdf?sequence=1

\section{Metody pro obcházení captive portálů}

% alfa-omega - pro pristup ke captive portalu musim byt validnim ucatnikem site
% tj musi fungovat DHCP (musim mit IP, musim zant GW), musi ``fungovat`` DNS
% to je problem zejo, kdyz uz na siti jsem je mnohem tezsi snazit se me nejak omezit
%
% jak se to da obejit.
% DNS tunely    DNSX, iodine (Ip Over DNS - atomic(I)=53 .. haha asociace)
% ICMP tunely   ICMPX
% VoIP SIP porty obcas otevrene
% web caching proxy 3128.
% mozna dokonce SSH 22?

% cilem prace je vyuzit co nejvic technik a delat MUX trafficu



% SSH spojeni, MUX (parsuje TCP stream a prenasi jen data po TCP(via SSH tunel)
% pro muj ucel nepouzitelne, ale zajiamve
% https://github.com/apenwarr/sshuttle


% pokud budu diskutovat kmitocty ISM pasma, ČTÚ ma na webu hezky popsany smernice odkud co a proc je bezlicencni atd.


\chapter{Návrh řešení}

Doplňte vhodný text.

\chapter{Implementace}

Doplňte vhodný text.

\chapter{Testování}

Doplňte vhodný text.


\begin{conclusion}
	Doplňte závěr.
	
\end{conclusion}

\bibliographystyle{csn690}
\bibliography{literatura}

\appendix

\chapter{Seznam použitých zkratek}
% \printglossaries



%TODO sort a-z
\begin{description}
	\item[DNS] Domain Name System
	\item[ICMP] Internet Control Message Protocol
	\item[XML] Extensible markup language
	\item[ISM] Industrial, Scientific and Medical radio bands
	\item[NAC] Network Access Control -- řízení síťového přístupu
	\item[GSM] Global System for Mobile Communications
	\item[MAC] Media Access Control
	\item[IP] Internet Protocol
	\item[MITM] Man-in-the-middle
	\item[HTTP] Hypertext Transfer Protocol
	\item[HTTPS] HTTP Secure
	\item[TTL] Time to live
	\item[VLAN] Virtual local area network
\end{description}


% LaTeX notes:

% citace: viz \cite{kobltypo}

% vlna

% \begin{figure}\centering
% 	\includegraphics[width=0.5\textwidth, angle=30]{cvut-logo-bw}
% 	\caption[Příklad obrázku]{Ukázkový obrázek v~plovoucím prostředí}\label{fig:float}
% \end{figure}

%Může se hodit mít více verzí stejného obrázku, např. pro barevný či černobílý tisk a nebo pro prezentaci.

% label -> (viz obr. \ref{fig:gnuplot-bw}) 

% \subsection{Tabulky}
% 
% Tabulky lze zadávat různě, např. v~prostředí \verb|tabular|, avšak pro jejich vkládání platí to samé, co pro obrázky -- použijte plovoucí prostředí, v~tomto případě \verb|table|. Například tabulka \ref{tab:matematika} byla vložena tímto způsobem.
% 
% \begin{table}\centering
% 	\caption[Příklad tabulky]{Zadávání matematiky}\label{tab:matematika}
% 	\begin{tabular}{|l|l|c|c|}\hline
% 		Typ		& Prostředí		& \LaTeX{}ovská zkratka	& \TeX{}ovská zkratka	\tabularnewline \hline \hline
% 		Text		& \verb|math|		& \verb|\(...\)|	& \verb|$...$|		\tabularnewline \hline
% 		Displayed	& \verb|displaymath|	& \verb|\[...\]|	& \verb|$$...$$|	\tabularnewline \hline
% 	\end{tabular}
% \end{table}


% vkládání URL a podobných informací url{} ze stejnojmenného balíčku. Zajistíte tím jednak odlišení adresy od ostatního textu pomocí jiného písma a také zalamování na konci řádku.
% Chcete-li vkládat odkazy (funkční v~PDF), použijte příkaz \verb|href| z~balíčku \verb|hyperref|.

% % % % % % % % % % % % % % % % % % % % % % % % % % % 

\chapter{Obsah přiloženého CD}


% Vhodným způsobem vizualizujte obsah přiloženého média. Lze použít balíček \verb|dirtree| a vytvořit např. následující výstup (adresáře src a text s~příslušným obsahem jsou \emph{povinné}):
% 
% \begin{figure}
% 	\dirtree{%
% 		.1 readme.txt\DTcomment{stručný popis obsahu CD}.
% 		.1 exe\DTcomment{adresář se spustitelnou formou implementace}.
% 		.1 src.
% 		.2 impl\DTcomment{zdrojové kódy implementace}.
% 		.2 thesis\DTcomment{zdrojová forma práce ve formátu \LaTeX{}}.
% 		.1 text\DTcomment{text práce}.
% 		.2 thesis.pdf\DTcomment{text práce ve formátu PDF}.
% 	}
% \end{figure}


\end{document}
